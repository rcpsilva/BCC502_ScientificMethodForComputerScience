\documentclass{article}
\usepackage[utf8]{inputenc}
\usepackage[margin=1.2in]{geometry}
\usepackage{hyperref}
\usepackage{listings}
\usepackage{xcolor}
\usepackage{natbib}
\usepackage{graphicx}
\usepackage{amsmath}
\usepackage{hyperref}

\title{\vspace{-2 cm}Universidade Federal de Ouro Preto \\ Trabalho Final}
\author{Prof. Rodrigo Silva}
\date{}

\begin{document}

\maketitle

\section{Leitura}

\begin{itemize}
    \item Capítulo 9 - WAZLAWICK, Raul Sidnei. Metodologia de pesquisa para ciência da computação. 3. ed. Rio de Janeiro: Elsevier, 2014.
\end{itemize}

\section{Questões}

\begin{enumerate}
\item Utilizando o template em LaTeX disponível no repositório da disciplina, cada aluno deve apresentar uma proposta de pesquisa contendo os seguintes itens:

\begin{enumerate}
    \item Contextualização e definição do problema de pesquisa.
    \item Objetivo geral e específicos.
    \item Definição e justificativa da hipótese de trabalho.
    \item Procedimento metodológico.
\end{enumerate} 

\end{enumerate}


%\bibliographystyle{plain}
%\bibliography{references}
\end{document}

