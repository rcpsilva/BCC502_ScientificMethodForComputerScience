\documentclass{article}
\usepackage[utf8]{inputenc}
\usepackage[margin=1.2in]{geometry}
\usepackage{hyperref}
\usepackage{listings}
\usepackage{xcolor}
\usepackage{natbib}
\usepackage{graphicx}
\usepackage{amsmath}

\title{\vspace{-2 cm}Universidade Federal de Ouro Preto \\ Computação e Classificação das Ciências}
\author{Prof. Rodrigo Silva}
\date{}


\begin{document}

\maketitle

\section{Leitura}

\begin{itemize}
    \item Capítulos 1 e 2 - WAZLAWICK, Raul Sidnei. Metodologia de pesquisa para ciência da computação. 3. ed. Rio de Janeiro: Elsevier, 2014.
\end{itemize}

\section{Questões}

\subsection{Capítulo 1}

\begin{enumerate}

\item Qual problema o aluno de mestrado decidiu atacar?

\item O que o aluno estudou primeiramente?

\item Em qual solução o aluno começou a trabalhar?  

\item Como foi o experimento projetado pelo aluno para testar a solução proposta?

\item Quais foram os resultados atingidos pelo aluno? Estes resultados são bons? 

\item Quais foram os erros cometidos pelo aluno?

\end{enumerate}

\subsection{Capítulo 2}

\begin{enumerate}

\item O que é uma ciência formal? Dê exemplos de ciências formais na subárea da computação.
\item O que é uma ciência empírica? O que é obrigatório neste tipo de ciência?
\item Diferencie as ciências naturais das ciências sociais. Quais displinas representam estas ciências na computação?
\item O que é computação científica?
\item O que é uma ciência pura?
\item O que é uma ciência aplicada?
\item Qual a diferença entre ciência exata e ciência inexata? Dê exemplos das duas.
\item Qual a diferença entre uma ciência hard e uma ciência soft? A Ciência da Computação é uma ciência har ou soft?
\item O que é uma ciência nomotética? E uma ciência idiográfica?

\end{enumerate}


%\bibliographystyle{plain}
%\bibliography{references}
\end{document}

