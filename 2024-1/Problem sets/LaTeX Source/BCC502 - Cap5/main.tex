\documentclass{article}
\usepackage[utf8]{inputenc}
\usepackage[margin=1.2in]{geometry}
\usepackage{hyperref}
\usepackage{listings}
\usepackage{xcolor}
\usepackage{natbib}
\usepackage{graphicx}
\usepackage{amsmath}

\title{\vspace{-2 cm}Universidade Federal de Ouro Preto \\ Maturidade em Pesquisa}
\author{Prof. Rodrigo Silva}
\date{}


\begin{document}

\maketitle

\section{Leitura}

\begin{itemize}
    \item Capítulo 5 - WAZLAWICK, Raul Sidnei. Metodologia de pesquisa para ciência da computação. 3. ed. Rio de Janeiro: Elsevier, 2014.
\end{itemize}

\section{Questões}

\begin{enumerate}
\item Quais são os principais tipos de pesquisa em ciência da comutação? Descreva cada um deles.
\item Quais as fraquezas de um trabalho do tipo ``Apresentação de um produto'? Por que este tipo de trabalho não tem  muito reconhecimento? Em que tipo de cenário este tipo de trabalho pode ser adequado?
\item Por que um trabalho do tipo ``Apresentação de algo diferente'' precisa de uma boa hipótese? 
\item Por que a tabela comparative é uma ferramenta útil em um trabalho do tipo ``Apresentação de algo diferente''? Quais os pontos fundamentais de uma tabela como essa?
\item Qual a diferença entre os tipos ``Apresentação de algo presumivelmente melhor'' e ``Apresentação de algo reconhecidamente melhor''?
\item Qual a importância da métrica de qualidade nos tipos de pesquisa ``Apresentação de algo presumivelmente melhor'' e ``Apresentação de algo reconhecidamente melhor''? Como ela deve ser definida?
\item Por que o tipo de pesquisa ``Apresentação de algo reconhecidamente melhor'' tende a ser mais fácil de executar? Qual a maior dificuldade deste tipo de pesquisa?
\item Quais são os três tipos básicos de pesquisa nos quais podemos enquadrar os tipos de pesquisa em ciência da computação? Quais as principais ferramentas utilizadas em cada um deles? Quais as fragilidades de cada tipo?

\end{enumerate}


%\bibliographystyle{plain}
%\bibliography{references}
\end{document}

