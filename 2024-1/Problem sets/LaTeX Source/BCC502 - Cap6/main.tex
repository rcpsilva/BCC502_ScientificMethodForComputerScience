\documentclass{article}
\usepackage[utf8]{inputenc}
\usepackage[margin=1.2in]{geometry}
\usepackage{hyperref}
\usepackage{listings}
\usepackage{xcolor}
\usepackage{natbib}
\usepackage{graphicx}
\usepackage{amsmath}

\title{\vspace{-2 cm}Universidade Federal de Ouro Preto \\ Elementos da Pesquisa}
\author{Prof. Rodrigo Silva}
\date{}

\begin{document}

\maketitle

\section{Leitura}

\begin{itemize}
    \item Capítulo 6 - WAZLAWICK, Raul Sidnei. Metodologia de pesquisa para ciência da computação. 3. ed. Rio de Janeiro: Elsevier, 2014.
\end{itemize}

\section{Observações}

Neste estudo dirigido já irei provocá-los a colocar no papel o que você planejam fazer como projeto de pesquisa (o trabalho final da disciplina). Nesse sentido, antes de responder às perguntas, façam uma leitura atenta de todo o capítulo 6 do livro-texto da disciplina.

Como vocês verão no texto, para responder às perguntas abaixo com qualidade, seria necessário fazer uma boa revisão de literatura. No entanto, neste momento, ainda não espero isso. Tentem ler artigos relacionados, mas não precisam fazer uma revisão de literatura aprofundada (por enquanto).

\section{Questões}

\begin{enumerate}
\item Qual será o tema do seu trabalho?
\item Como este tema está relacionado com a sua perspectiva de desenvolvimento profissional?
\item Qual o seu problema de pesquisa?
\item Defina o objetivo do seu projeto de pesquisa.
\item Defina os objetivos específicos do seu projeto de pesquisa.
\item Por que este objetivo é relevante? Por que vale a pena perseguir este objetivo?
\item O seu objetivo foi definido em termos de definições constitutivas ou definições operacionais?
\item Descreva brevemente um procedimento metodológico para demonstar que o seu objetivo foi concluído.

\end{enumerate}


%\bibliographystyle{plain}
%\bibliography{references}
\end{document}

