\documentclass{article}
\usepackage{amsmath}
\usepackage[margin=1.2cm]{geometry}

\begin{document}

\title{ Questões sobre o Capítulo 7}
\author{}
\date{}
\maketitle

\begin{enumerate}

% Questão 1
\item Qual é o principal objetivo de uma revisão de literatura em um projeto de pesquisa?
\begin{enumerate}
    \item Confirmar a hipótese de pesquisa
    \item Identificar lacunas no conhecimento existente
    \item Coletar dados primários para o estudo
    \item Definir a metodologia da pesquisa
\end{enumerate}

\item Em uma revisão de literatura bem-sucedida, qual das seguintes práticas pode comprometer a imparcialidade e a integridade da análise?
\begin{enumerate}
\item Selecionar fontes que apresentam tanto resultados positivos quanto negativos sobre o tema de pesquisa
\item Incluir apenas estudos que corroboram a hipótese inicial do pesquisador
\item Avaliar criticamente a metodologia e os resultados de cada estudo selecionado
\item Estruturar a revisão de forma lógica e coerente, integrando diferentes perspectivas
\end{enumerate}

\item Por que é importante que o pesquisador mantenha uma atitude crítica ao ler trabalhos científicos, segundo o texto?
\begin{enumerate}
\item Para aceitar todas as informações como verdadeiras sem questionamento
\item Para transformar a leitura em uma fonte de questionamentos que possam gerar novos objetivos de pesquisa
\item Para acelerar o processo de leitura e compreensão dos trabalhos científicos
\item Para identificar rapidamente as respostas às perguntas já existentes
\end{enumerate}

\item Qual é a relevância da pergunta "O que foi feito poderia ser feito de outro modo?" ao avaliar um trabalho científico?
\begin{enumerate}
\item Verificar se o trabalho pode ser replicado em outras áreas de pesquisa
\item Avaliar se a abordagem escolhida pelo autor foi a mais adequada e se existem alternativas que poderiam ser exploradas
\item Garantir que o trabalho tenha sido realizado com uma amostra suficientemente grande
\item Confirmar que os resultados foram aplicados corretamente pelo autor
\end{enumerate}

\item Como a pergunta "Que ideias de outras áreas poderiam ser aproveitadas nesse trabalho?" pode contribuir para o desenvolvimento de novas pesquisas?
\begin{enumerate}
\item Sugere a aplicação de conceitos de áreas correlatas para potencialmente melhorar o trabalho estudado
\item Determina se os resultados do trabalho são aplicáveis apenas à área específica de estudo
\item Identifica as fraquezas metodológicas do trabalho em questão
\item Estabelece as limitações do trabalho dentro de sua área original
\end{enumerate}

\item Por que é importante que alunos de mestrado e doutorado tentem gerar ideias de pesquisa regularmente para discutir com seus orientadores?
\begin{enumerate}
\item Para acelerar a conclusão de sua dissertação ou tese
\item Para manter um fluxo contínuo de ideias que podem ser refinadas e transformadas em objetivos de pesquisa
\item Para garantir que suas ideias sejam publicadas rapidamente
\item Para evitar a necessidade de participar de palestras e seminários
\end{enumerate}

\pagebreak

\item Qual é o benefício de participar regularmente de defesas de teses e dissertações, mesmo que não sejam da área específica de pesquisa do aluno?
\begin{enumerate}
\item Melhorar a capacidade de realizar revisões sistemáticas
\item Aprender como os trabalhos são avaliados pelas bancas e compreender diferentes abordagens de pesquisa
\item Reduzir o tempo necessário para escrever a própria tese ou dissertação
\item Aumentar a quantidade de publicações em periódicos de alto impacto
\end{enumerate}

\item Qual é a principal diferença entre uma revisão de literatura sistemática e uma revisão narrativa?
\begin{enumerate}
    \item  A revisão narrativa é mais objetiva que a sistemática
    \item  A revisão sistemática segue uma metodologia predefinida e rigorosa, a narrativa é mais flexível e interpretativa
    \item  A revisão sistemática é usada apenas em ciências exatas, enquanto a narrativa é usada em ciências humanas
    \item  A revisão narrativa é mais detalhada que a sistemática
\end{enumerate}

\item Qual é a principal função do protocolo de revisão em uma revisão sistemática?
\begin{enumerate}
\item Definir o cronograma de trabalho para a equipe de pesquisadores
\item Estabelecer um documento formal que detalha todo o processo de revisão para evitar desvios durante o estudo
\item Identificar os artigos mais relevantes para serem incluídos na revisão
\item Publicar os resultados da revisão em sites especializados
\end{enumerate}

\item Qual das seguintes práticas é recomendada para mitigar problemas durante a execução de uma revisão sistemática, conforme descrito no protocolo de revisão?
\begin{enumerate}
\item Evitar a revisão-piloto para economizar tempo e recursos
\item Analisar apenas os primeiros 50 artigos retornados pela string de busca
\item Realizar uma revisão-piloto com escopo reduzido para identificar possíveis ajustes antes de iniciar a revisão completa
\item Incluir todos os artigos retornados pela busca, independentemente do volume
\end{enumerate}

\item Por que é importante verificar se já existe uma revisão sistemática publicada sobre o tema de interesse antes de iniciar sua própria revisão?
\begin{enumerate}
\item Para garantir que seu trabalho seja o primeiro publicado na área
\item Para evitar duplicação de esforços e utilizar uma revisão existente que já responda às suas perguntas de pesquisa
\item Para obter acesso exclusivo aos portais de revisão sistemática
\item Para assegurar que todos os estudos primários foram incluídos na nova revisão
\end{enumerate}

\item Quando é mais apropriado optar por um mapeamento sistemático em vez de uma revisão sistemática, segundo o texto?
\begin{enumerate}
\item Quando o objetivo é responder a uma questão de pesquisa específica
\item Quando há um número limitado de estudos primários disponíveis
\item Quando o objetivo é explorar a extensão de um tópico de pesquisa de forma ampla
\item Quando se deseja repetir uma revisão já existente com novos critérios
\end{enumerate}

\item Qual é um dos principais riscos associados ao uso de uma string de busca muito genérica em uma revisão sistemática?
\begin{enumerate}
\item  Aumenta a relevância dos artigos encontrados
\item  Diminui o número de estudos retornados
\item  Pode resultar em um número muito grande de estudos, dificultando a conclusão da pesquisa
\item  Garante que todos os trabalhos relevantes sejam incluídos
\end{enumerate}

\item Como pode-se mitigar o risco de deixar de fora trabalhos importantes devido a uma string de busca muito restritiva?
\begin{enumerate}
\item  Adicionando mais termos genéricos à string de busca
\item  Usando apenas termos em inglês
\item  Listando trabalhos conhecidos como relevantes e verificando se eles são retornados nas buscas
\item  Reduzindo o número de sinônimos na string de busca
\end{enumerate}

\item Qual é o problema associado ao viés quando se combinam dados de várias publicações para criar uma base estatística maior?
\begin{enumerate}
\item  Aumenta a diversidade de resultados
\item  Pode criar um conjunto de dados enviesado, levando a conclusões possivelmente erradas
\item  Garante que todos os estudos, independentemente de seus resultados, sejam considerados
\item  Reduz a relevância dos estudos incluídos na análise
\end{enumerate}

% Questão 10
\item Qual é a importância da revisão por pares na seleção de artigos para uma revisão de literatura?
\begin{enumerate}
    \item  Garante que todos os artigos sejam de acesso aberto
    \item  Assegura a qualidade e a credibilidade dos estudos revisados
    \item  Facilita a síntese dos resultados
    \item  Aumenta o número de referências disponíveis
\end{enumerate}

% Questão 11
\item Por que a avaliação crítica das fontes é crucial em uma revisão de literatura?
\begin{enumerate}
    \item  Para garantir que todas as fontes citadas são recentes
    \item  Para evitar a inclusão de estudos com falhas metodológicas graves
    \item  Para reduzir o número de referências a serem utilizadas
    \item  Para identificar as lacunas de pesquisa de forma mais eficaz
\end{enumerate}

% Questão 19
\item Por que é importante considerar o ano de publicação na seleção de artigos para a revisão de literatura?
\begin{enumerate}
    \item  Estudos mais antigos sempre são menos relevantes
    \item  Para garantir que a revisão reflita o estado atual do conhecimento
    \item  Para incluir apenas estudos publicados na última década
    \item  Porque os estudos mais recentes são sempre de maior qualidade
\end{enumerate}

% Questão 20
\item Como a identificação de palavras-chave afeta a revisão de literatura?
\begin{enumerate}
    \item  Ela determina a quantidade de artigos que serão revisados
    \item  Ela influencia diretamente a abrangência e a relevância dos estudos selecionados
    \item  Ela é irrelevante se a revisão incluir todos os artigos disponíveis
    \item  Ela é utilizada apenas para a organização da bibliografia
\end{enumerate}

\item Qual é o principal objetivo de avaliar a qualidade dos estudos primários em uma revisão sistemática?
\begin{enumerate}
\item Confirmar a originalidade dos estudos incluídos
\item Estabelecer um ponto de corte para remover estudos cuja qualidade seja considerada insuficiente
\item Determinar a relevância estatística dos resultados
\item Identificar novas áreas de pesquisa
\end{enumerate}

\item  Qual das seguintes perguntas é mais relevante na avaliação da qualidade de estudos qualitativos, segundo Kitchenham e Charters (2007)?
\begin{enumerate}
\item O tamanho da amostra era adequado?
\item A análise dos dados foi apropriada?
\item As descobertas são críveis?
\item A seleção dos elementos a serem estudados foi feita de forma aleatória?
\end{enumerate}

\item Como a análise de sensibilidade é utilizada na avaliação dos resultados da extração de dados em uma revisão sistemática?
\begin{enumerate}
\item Para determinar se os resultados foram confirmados por técnicas estatísticas adequadas
\item Para verificar se a subdivisão do conjunto de estudos em subconjuntos apresenta resultados consistentes com o conjunto completo
\item Para avaliar a clareza das ligações entre os dados e as conclusões
\item Para garantir que a coleta de dados foi realizada de maneira precisa
\end{enumerate}

\end{enumerate}

\end{document}
