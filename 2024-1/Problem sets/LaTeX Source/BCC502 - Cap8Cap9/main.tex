\documentclass{article}
\usepackage[utf8]{inputenc}
\usepackage[margin=1.2in]{geometry}
\usepackage{hyperref}
\usepackage{listings}
\usepackage{xcolor}
\usepackage{natbib}
\usepackage{graphicx}
\usepackage{amsmath}

\title{\vspace{-2 cm}Universidade Federal de Ouro Preto \\ Contextualização, Problema de Pesquisa e Objetivo}
\author{Prof. Rodrigo Silva}
\date{}

\begin{document}

\maketitle

\section{Leitura}

\begin{itemize}
    \item Capítulo 8 - WAZLAWICK, Raul Sidnei. Metodologia de pesquisa para ciência da computação. 3. ed. Rio de Janeiro: Elsevier, 2014.
    \item Capítulo 9 - WAZLAWICK, Raul Sidnei. Metodologia de pesquisa para ciência da computação. 3. ed. Rio de Janeiro: Elsevier, 2014.
\end{itemize}

\section{Questões}

\begin{enumerate}
\item Após a leitura dos capítulos indicados, cada aluno deverá contextualizar e apresentar o problema de pesquisa, bem como o objetivo geral de seu trabalho em uma breve apresentação de slides (de aproximadamente 2 ou 3 slides). As apresentações devem ser enviadas pelo moodle até o dia 20/08/2024 às 17h, e não serão aceitas após esse prazo. Alguns alunos serão selecionados aleatoriamente para apresentar seus slides à turma na próxima aula.
\end{enumerate}


%\bibliographystyle{plain}
%\bibliography{references}
\end{document}

