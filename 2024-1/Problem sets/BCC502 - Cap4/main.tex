\documentclass{article}
\usepackage[utf8]{inputenc}
\usepackage[margin=1.2in]{geometry}
\usepackage{hyperref}
\usepackage{listings}
\usepackage{xcolor}
\usepackage{natbib}
\usepackage{graphicx}
\usepackage{amsmath}

\title{\vspace{-2 cm}Universidade Federal de Ouro Preto \\ Tipos de Pesquisa}
\author{Prof. Rodrigo Silva}
\date{}


\begin{document}

\maketitle

\section{Leitura}

\begin{itemize}
    \item Capítulo 4 - WAZLAWICK, Raul Sidnei. Metodologia de pesquisa para ciência da computação. 3. ed. Rio de Janeiro: Elsevier, 2014.
\end{itemize}

\section{Questões}

\begin{enumerate}
\item Quais são as possíveis classificações de uma pesquisa quanto à sua natureza? Explique cada classificação.
\item Por quê recomenda-se que uma pesquisa primária seja precedida de um mapeamento da literatura?   
\item Quais são as classificações de uma pesquisa quanto ao seu objetivo? Qual a diferença entre elas? Qual delas é reconhecida como de excelência? Por quê?
\item Defina pesquisa bibliográfica.
\item Defina pesquisa experimental.
\item Defina pesquisa não experimental.
\item O que é uma pesquisa de levantamento? Qual o seu maior desafio?
\item Defina pesquisa etnográfica.
\item O que é um estudo de caso? Como ele pode ser dividido?
\item Qual a diferença entre ciência e tecnologia? 
\item Como podemos determinar se um trabalho tem cunho científico?
\item Usualmente, como é construído um trabalho científico?
\item O que a pesquisa em Ciência da Computação deve buscar?
\item Sobre o quê se estrutura um trabalho científico?
\item Quais os passos que geralmente levam a uma pesquisa bem sucedida? 
\end{enumerate}


%\bibliographystyle{plain}
%\bibliography{references}
\end{document}

