\documentclass{article}
\usepackage[utf8]{inputenc}
\usepackage[margin=1.2in]{geometry}
\usepackage{hyperref}
\usepackage{listings}
\usepackage{xcolor}
\usepackage{natbib}
\usepackage{graphicx}
\usepackage{amsmath}

\title{\vspace{-2 cm}Universidade Federal de Ouro Preto \\ Inteligência Artificial \\ Estudo Dirigido 1}
\author{Prof. Rodrigo Silva}
\date{}


\begin{document}

\maketitle

\section{Leitura}

\begin{itemize}
    \item Capítulos 1 e 2 - WAZLAWICK, Raul Sidnei. Metodologia de pesquisa para ciência da computação. 2. ed. Rio de Janeiro: Elsevier, 2014.
\end{itemize}

\section{Questões}

\begin{enumerate}

\item Quais foram os erros cometidos pelo ``aluno de mestrado'' citado no capítulo 1?

\item Quais são os estilos de pesquisa em computação?

\item Desenvolvimento de software pode ser considerado pesquisa científica? Se sim, como este tipo de pesquisa deve ser conduzido?

\item O que é uma hipótese? Por quê ela é importante?

\item O que é o tema de pesquisa?

\end{enumerate}


%\bibliographystyle{plain}
%\bibliography{references}
\end{document}

