\documentclass{article}
\usepackage[utf8]{inputenc}
\usepackage[margin=1.2in]{geometry}
\usepackage{hyperref}
\usepackage{listings}
\usepackage{xcolor}
\usepackage{natbib}
\usepackage{graphicx}
\usepackage{amsmath}

\title{\vspace{-2 cm}Universidade Federal de Ouro Preto \\  Método Científico}
\author{Prof. Rodrigo Silva}
\date{}


\begin{document}

\maketitle

\section{Leitura}

\begin{itemize}
    \item Capítulo 3- WAZLAWICK, Raul Sidnei. Metodologia de pesquisa para ciência da computação. 3. ed. Rio de Janeiro: Elsevier, 2014.
\end{itemize}

\section{Questões}

\begin{enumerate}
\item Qual a relação entre teoria e coleta de dados no método científico?
\item No contexto científico, o que significa a palavra \textit{empírico}?
\item Como um empiricista pensa a ciência? 
\item O que precisaríamos fazer para que a afirmação ``a maioria dos programadores não gosta de usar UML'' tivesse embasamento empírico?
\item Qual a diferença entre pragmatismo e realismo? Com qual destas duas você se indentifica mais? Explique.
\item Qual é a crítica feita em relação ao ``conhecimento absoluto''?
\item Defina o princípio da objetividade. Você acha que a pesquisa em computação deveria seguir o princípio da objetividade? Você consegue estabeler alguma conexão entre objetividade e pragmatismo?
\item Explique como o princípio da indução é utilizado na ciência empírica?
\item Descreva o princípio da refutação.
\item Descreva o princípio do coerentismo.
\item O que é o princípio da Lâmina de Occam?
\item Contraponha o construtivismo e o reducionismo? Quais as diferenças entre os dois? Qual princípio você acha o mais adequado?
\end{enumerate}


%\bibliographystyle{plain}
%\bibliography{references}
\end{document}

