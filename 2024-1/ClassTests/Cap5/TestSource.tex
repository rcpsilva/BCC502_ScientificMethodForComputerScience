\documentclass[a4paper,12pt]{article}
\usepackage[brazil]{babel}
\usepackage[utf8]{inputenc}
\usepackage{enumitem}
\usepackage{geometry}

\geometry{
  a4paper,
  left=2cm,
  right=2cm,
  top=2cm,
  bottom=2cm
}

\title{Capítulo 4 - Tipos de Pesquisa}
\author{}
\date{}

\begin{document}

\maketitle

\vspace{-2cm}

\footnotesize

\begin{enumerate}

    \item Qual é o objetivo principal da pesquisa primária?
    \begin{enumerate}[label=(\alph*)]
        \item Obter informações a partir de trabalhos já publicados.
        \item Realizar mapeamento sistemático da literatura.
        \item Apresentar conhecimento novo a partir de observações e teorias.
        \item Realizar revisão sistemática sobre revisões sistemáticas.
    \end{enumerate}

\item O que caracteriza a pesquisa secundária?
    \begin{enumerate}[label=(\alph*)]
        \item Realização de experimentos para descobrir novas informações.
        \item Coleta de dados inéditos através de observações.
        \item Obtenção de informações a partir de trabalhos já publicados.
        \item Manipulação de variáveis para testar hipóteses.
    \end{enumerate}

\item Por que é recomendado que uma pesquisa primária seja precedida por um mapeamento sistemático da literatura?
    \begin{enumerate}[label=(\alph*)]
        \item Para garantir que a pesquisa seja a primeira de seu tipo.
        \item Para ajudar a formular boas questões de pesquisa e verificar se já foram respondidas na literatura.
        \item Para substituir a necessidade de experimentos e observações.
        \item Para produzir gráficos estatísticos com conclusões sólidas.
    \end{enumerate}

\item Quando é justificada a realização de uma pesquisa terciária?
    \begin{enumerate}[label=(\alph*)]
        \item Quando a pesquisa primária não apresenta resultados satisfatórios.
        \item Quando se deseja explorar uma área de conhecimento pouco estudada.
        \item Quando há um número significativo de pesquisas secundárias publicadas.
        \item Quando se quer criar novas teorias a partir de dados empíricos.
    \end{enumerate}

    \item Qual é a característica principal da pesquisa exploratória?
    \begin{enumerate}[label=(\alph*)]
        \item Analisar os dados observados para buscar suas causas e explicações.
        \item Descrever os fatos como eles são ou categorizá-los.
        \item Examinar fenômenos sem uma hipótese ou objetivo definido em mente.
        \item Determinar como as coisas poderiam ser, propondo soluções ideais.
    \end{enumerate}

\item O que diferencia a pesquisa descritiva da pesquisa exploratória?
    \begin{enumerate}[label=(\alph*)]
        \item A pesquisa descritiva busca causas e explicações, enquanto a exploratória propõe soluções ideais.
        \item A pesquisa descritiva procura obter dados consistentes sem interferência, enquanto a exploratória busca anomalias.
        \item A pesquisa descritiva é o primeiro estágio de um processo de pesquisa, enquanto a exploratória é um passo prévio para encontrar fenômenos não explicados.
        \item A pesquisa descritiva realiza experimentos, enquanto a exploratória analisa dados observados.
    \end{enumerate}

\item Qual é o objetivo principal da pesquisa explicativa?
    \begin{enumerate}[label=(\alph*)]
        \item Identificar anomalias conhecidas ou não em um conjunto de fenômenos.
        \item Obter dados consistentes sobre determinada realidade.
        \item Analisar dados observados para buscar causas e explicações dos fenômenos.
        \item Determinar como as coisas poderiam ser, criando soluções ideais.
    \end{enumerate}

\item Quando uma pesquisa de design é considerada relevante?
    \begin{enumerate}[label=(\alph*)]
        \item Quando há necessidade de descrever fatos como eles são.
        \item Quando se busca determinar como as coisas poderiam ser, propondo soluções ideais.
        \item Quando é necessário obter dados consistentes sem interferência.
        \item Quando se deseja analisar os dados observados para buscar causas e explicações.
    \end{enumerate}

\item Qual das alternativas melhor descreve a pesquisa descritiva?
    \begin{enumerate}[label=(\alph*)]
        \item Examinar fenômenos sem hipótese definida, buscando anomalias.
        \item Obter dados consistentes sobre determinada realidade sem interferência do pesquisador.
        \item Analisar dados observados para buscar suas causas e explicações.
        \item Determinar como as coisas poderiam ser, propondo soluções ideais.
    \end{enumerate}

    \item Qual é o principal objetivo da pesquisa bibliográfica?
    \begin{enumerate}[label=(\alph*)]
        \item Estudar artigos, teses, livros e outras publicações indexadas.
        \item Analisar documentos ou dados não sistematizados e publicados.
        \item Manipular variáveis experimentais para obter novos dados.
        \item Realizar questionários e entrevistas com grupos de pessoas.
    \end{enumerate}

\item O que caracteriza a pesquisa documental?
    \begin{enumerate}[label=(\alph*)]
        \item Estudo de fenômenos sem a intervenção do pesquisador.
        \item Análise de documentos ou dados não sistematizados e publicados.
        \item Manipulação de variáveis experimentais para testar hipóteses.
        \item Observação direta de comportamentos em um grupo social.
    \end{enumerate}

\item Quais são as principais dificuldades encontradas na realização de pesquisa experimental em Computação?
    \begin{enumerate}[label=(\alph*)]
        \item Coleta de dados e realização de entrevistas.
        \item Manipulação ou medição de variáveis e o tempo que as intervenções podem levar.
        \item Análise de documentos não sistematizados e publicados.
        \item Observação de fenômenos sem intervenção sistemática.
    \end{enumerate}

\item Qual é o maior desafio para a pesquisa de levantamento em relação ao viés da amostra?
    \begin{enumerate}[label=(\alph*)]
        \item Determinar como as variáveis experimentais afetam os resultados.
        \item Coletar dados consistentes sem interferência do pesquisador.
        \item Evitar que as respostas dos participantes sejam influenciadas por suas relações com o objeto de estudo.
        \item Observar os fenômenos diretamente no ambiente natural sem intervenção.
    \end{enumerate}

\item Como a pesquisa-ação difere das outras formas de pesquisa em Computação?
    \begin{enumerate}[label=(\alph*)]
        \item Ela se baseia na análise de documentos não sistematizados.
        \item Ela envolve o pesquisador de forma participativa, buscando resolver problemas ainda não resolvidos.
        \item Ela utiliza técnicas de amostragem e testes de hipóteses para resultados estatisticamente aceitáveis.
        \item Ela foca na descrição de fenômenos sem interferência do pesquisador.
    \end{enumerate}

    \item Qual é o principal objetivo da ciência?
    \begin{enumerate}[label=(\alph*)]
        \item Aplicar conhecimentos em atividades práticas.
        \item Construir teorias para explicar fatos observados.
        \item Desenvolver sistemas e protótipos.
        \item Transformar o mundo através de atividades industriais.
    \end{enumerate}

\item O que diferencia ciência de tecnologia, segundo o texto?
    \begin{enumerate}[label=(\alph*)]
        \item A ciência busca transformar o mundo, enquanto a tecnologia busca explicá-lo.
        \item A ciência é prática e aplicada, enquanto a tecnologia é teórica.
        \item A ciência busca conhecimento e explicações, enquanto a tecnologia aplica esses conhecimentos em atividades práticas.
        \item A ciência envolve atividades industriais, enquanto a tecnologia é restrita a pesquisas acadêmicas.
    \end{enumerate}

\item Por que um trabalho em Computação pode ser considerado meramente tecnológico e não científico?
    \begin{enumerate}[label=(\alph*)]
        \item Porque ele apresenta sistemas, protótipos, frameworks e arquiteturas sem explicar por que ou como funcionam.
        \item Porque ele não utiliza o método científico.
        \item Porque ele não tem aplicação prática.
        \item Porque ele se baseia exclusivamente em revisão bibliográfica.
    \end{enumerate}

\item O que é necessário para que um trabalho seja considerado de cunho científico em Computação?
    \begin{enumerate}[label=(\alph*)]
        \item Que ele apresente novos sistemas e protótipos.
        \item Que ele explique um pouco mais sobre o porquê das coisas funcionarem como funcionam ou como poderiam funcionar melhor.
        \item Que ele se baseie exclusivamente em dados experimentais.
        \item Que ele utilize técnicas de amostragem e testes de hipóteses.
    \end{enumerate}

\item Qual é a importância de uma revisão bibliográfica no início de uma pesquisa científica?
    \begin{enumerate}[label=(\alph*)]
        \item Identificar os principais conceitos da área e os últimos desenvolvimentos.
        \item Coletar dados empíricos para testar a hipótese.
        \item Aplicar conhecimentos em atividades práticas.
        \item Realizar experimentos controlados e replicáveis.
    \end{enumerate}

\end{enumerate}

\newpage

\end{document}
