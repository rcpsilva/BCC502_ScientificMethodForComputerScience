\documentclass[a4paper,12pt]{article}
\usepackage[brazil]{babel}
\usepackage[utf8]{inputenc}
\usepackage{enumitem}

\title{Capítulo 4 - Tipos de Pesquisa}
\author{}
\date{}

\begin{document}

\maketitle

\vspace{-3cm}

\section*{Gabarito}

\begin{enumerate}

\item \textbf{(c)} Apresentar conhecimento novo a partir de observações e teorias.

\item \textbf{(c)} Obtenção de informações a partir de trabalhos já publicados.

\item \textbf{(b)} Para ajudar a formular boas questões de pesquisa e verificar se já foram respondidas na literatura.

\item \textbf{(c)} Quando há um número significativo de pesquisas secundárias publicadas.

\item \textbf{(c)} Examinar fenômenos sem uma hipótese ou objetivo definido em mente.

\item \textbf{(b)} A pesquisa descritiva procura obter dados consistentes sem interferência, enquanto a exploratória busca anomalias.

\item \textbf{(c)} Analisar dados observados para buscar causas e explicações dos fenômenos.

\item \textbf{(b)} Quando se busca determinar como as coisas poderiam ser, propondo soluções ideais.

\item \textbf{(b)} Obter dados consistentes sobre determinada realidade sem interferência do pesquisador.

\item \textbf{(a)} Estudar artigos, teses, livros e outras publicações indexadas.

\item \textbf{(b)} Análise de documentos ou dados não sistematizados e publicados.

\item \textbf{(b)} Manipulação ou medição de variáveis e o tempo que as intervenções podem levar.

\item \textbf{(c)} Evitar que as respostas dos participantes sejam influenciadas por suas relações com o objeto de estudo.

\item \textbf{(b)} Ela envolve o pesquisador de forma participativa, buscando resolver problemas ainda não resolvidos.

\item \textbf{(b)} Construir teorias para explicar fatos observados.

\item \textbf{(c)} A ciência busca conhecimento e explicações, enquanto a tecnologia aplica esses conhecimentos em atividades práticas.

\item \textbf{(a)} Porque ele apresenta sistemas, protótipos, frameworks e arquiteturas sem explicar por que ou como funcionam.

\item \textbf{(b)} Que ele explique um pouco mais sobre o porquê das coisas funcionarem como funcionam ou como poderiam funcionar melhor.

\item \textbf{(a)} Identificar os principais conceitos da área e os últimos desenvolvimentos.
    
\end{enumerate}

\end{document}
