\documentclass{article}
\usepackage[utf8]{inputenc}
\usepackage[margin=1.2in]{geometry}
\usepackage{hyperref}
\usepackage{listings}
\usepackage{xcolor}
\usepackage{natbib}
\usepackage{graphicx}
\usepackage{amsmath}

\title{\vspace{-2 cm}Universidade Federal de Ouro Preto \\ BCC502 - Metodologia Científica para Ciência da Computação \\\textbf{ Ciência como conhecimento derivado dos fatos da experiência}}
\author{Prof. Rodrigo Silva}
\date{}


\begin{document}

\maketitle

\section{Introdução}

\begin{enumerate}
    \item A ciência é altalmente considerada
    \begin{enumerate}
        \item Existe uma crença amplamente aceita de que há algo especial sobre especial
        \begin{enumerate}
            \item Dizer que algo é científico parece dar credibilidade
            \item ``Os cientistas dizem ...''
            \item Científicamente comprovado
        \end{enumerate}
        \item Insinuação que a afirmação é particularmente bem fundamentada e que talvez esteja além da contestação.
        \item Qual a base para tal autoridade? 
    \end{enumerate}
    \item O que é especial em relação à ciência? O que vem a ser este ``método científico'' que leva a resultados especialmente meritórios e confiáveis?
    \begin{enumerate}
        \item A filosofia da ciência tenta responder estas perguntas
        \item A história da ciência coloca muitos problemas para os filósofos. 
        \item Galileu, Newton, Darwin, Eisntein chegaram às suas contribuições científicas por caminhos e métodos muito diferentes.
    \end{enumerate}
    \item Existem críticos, e.g., Paul Feyerbend - Contra o método
    \begin{enumerate}
        \item A ciência é uma religião moderma. 
        \item Não possui características especiais que a façam superior a nada.
        \item Sugere que a escolha entre teorias se reduz a opções determinadas por valores subjetivos e e desejos pessoais 
    \end{enumerate}
    \item Positivismo lógigo
    \begin{enumerate}
        \item Afirma que apenas as declarações verificáveis através da observação direta ou da prova lógica são significativas.
        \item Conflito com a física quantíca e o com o relativismo de Eisntein.
    \end{enumerate}
    \item \textit{Começaremos confusos e terminaremos confusos num nível mais alto.}
\end{enumerate}

\section{Ciência como conhecimento derivado dos fatos da experiência}

\begin{enumerate}
    \item Uma visão comum da ciência
    \begin{enumerate}
        \item A ciência é derivada dos fatos.
        \begin{enumerate}
            \item Os fatos são presumidos como afirmações sobre o mundo que podem ser diretamente estabelecidas pelo uso cuidadoso e imparcial dos sentidos.
            \item A Ciência deve ser baseada no que podemos ver, ouvir, tocar.
            \item Não deve ser baseada nas nossa opiniões pessoais ou devaneios especulativos.
            \item Se a observação do mundo for realizada de maneira cuidadosa e imparcial, então os fatos estabelecidos dessa maneira constituirão uma base segura e objetiva para a ciência.
            \item Se o raciocínio que nos leva desse conjunto sólido de fatos à teorias e leis que constituem o conhecimento científico é sólido, conhecimento resultante pode ser considerado seguro e objetivo.  
        \end{enumerate}
        \item É comum encontrar a informação que a ciência moderna nasceu no início do século XVII, quando a estratégia de levar os fatos a observação e experimentação a sério como base para a ciência foi adotada pela primeira vez de forma significativa.
        \begin{enumerate}
            \item Antes, o conhecimento era baseado em grande parte na autoridade. A autoridade do filósofo Aristóteles e na autoridade da Bíblia. 
            \item Galileu (Pai da ciência moderna) - Modelo heliocêntrico, Leis do movimento para corpos em queda (Experimento da torre de Pisa).
        \end{enumerate}
    \end{enumerate}
    \item Duas escolas de pensamento tentaram formalizar essa ideia de ciência.
    \begin{enumerate}
        \item Empiricitas - Todo conhecimento deve ser derivado de ideias implantadas na mente por meio da percepção sensorial.
        \item Positivistas - O conhecimento deve ser derivado dos fatos da experiência. (menos psicológico)
        \item O empirismo e o positivismo compartilham a visão comum de que o conhecimento científico de alguma forma deve ser derivado dos fatos obtidos por meio da observação.
    \end{enumerate}
    \item Três premissas da posição de que os fatos são a base ciência:
    \begin{enumerate}
        \item Os fatos são diretamente fornecidos a observadores cuidadosos e imparciais por meio dos sentidos.
        \item Os fatos são anteriores e independentes da teoria.
        \item Os fatos constituem uma base sólida e confiável para o conhecimento científico.
    \end{enumerate}
    \item Os fatos são diretamente fornecidos por meio dos sentidos
    \begin{enumerate}
        \item Ilusões visuais
        \item O observador experiente e habilidoso não tem experiências perceptuais idênticas às do novato não treinado quando ambos enfrentam a mesma situação (Raio X, Bugs).
        \item Isso entra em conflito a afirmação de que as percepções são dadas de forma direta pelos sentidos.
        \item As imagens em nossas retinas determinam de forma única nossas experiências perceptuais?
    \end{enumerate}
    \item Fatos como descrições observacionais
    \begin{enumerate}
        \item Conhecimento é construído a partir de descrições e não dos fatos em si.
        \item Uma descrição depende do conhecimento do observador.
        \item Exemplo: Levantar fatos sobre a flora de um local
        \item Portanto, o registro de fatos observáveis requer mais do que a recepção dos estímulos.
        \item Novamente, temos um conflito com a ideia de que os fatos são diretamente fornecidos por meio dos sentidos.
        \item Declarações de fatos não são determinadas de maneira direta pelos estímulos sensoriais, e as declarações de observação pressupõem conhecimento, portanto, não pode ser o caso de que primeiro estabelecemos os fatos e depois derivamos nosso conhecimento deles.
    \end{enumerate}
    \item Os fatos constituem uma base sólida e confiável para o conhecimento científico.
    \begin{enumerate}
        \item A Terra está estacionária' não é estabelecida pela evidência observável da maneira como costumava ser pensado.
        \item Para compreender completamente por que isso é assim, precisamos entender a inércia.
    \end{enumerate}
    \item Conclusão
    \begin{enumerate}
        \item A ideia de que o conhecimento científico vem diretamente de fatos observáveis tem problemas
        \begin{enumerate}
            \item A descrição depende da experiência e conhecimento do observador
            \item A própria observação, coleta do fato, depende do observador.
            \item O julagamento de um fato observável depende do conhecimento contra o qual o julgamento é feito. (A não estacionaridade da terra diante da lei da inércia)
        \end{enumerate}
        \item Nem tudo está perdido
        \begin{enumerate}
            \item Mesmo que as as descrições observacionais dependam do observador, uma vez definidas, podem ser verificadas. Então existe uma ligação sólida entre a descrição observacional e o fato observado.
            \item Um erro sobre os fatos observáveis pode ser corrigido pelo conhecimento e tecnologia aprimorados (Tamanho de Venus e Marte vistos a olho nu).
        \end{enumerate}
        \item Isso mostra que qualquer visão de que o conhecimento científico se baseia nos fatos adquiridos pela observação deve permitir que os fatos, assim como o conhecimento, sejam falíveis e sujeitos a correções, e que o conhecimento científico e os fatos nos quais se poderia dizer que ele se baseia são interdependentes.
    \end{enumerate}
\end{enumerate}

\section*{Fonte}

\begin{itemize}
    \item Chalmers, Alan F. What is this thing called science?. Hackett Publishing, 2013.
\end{itemize}

%\bibliographystyle{plain}
%\bibliography{references}
\end{document}

