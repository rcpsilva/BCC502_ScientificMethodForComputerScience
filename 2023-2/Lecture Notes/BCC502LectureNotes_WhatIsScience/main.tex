\documentclass{article}
\usepackage[utf8]{inputenc}
\usepackage[margin=1.2in]{geometry}
\usepackage{hyperref}
\usepackage{listings}
\usepackage{xcolor}
\usepackage{natbib}
\usepackage{graphicx}
\usepackage{amsmath}

\title{\vspace{-2 cm}Universidade Federal de Ouro Preto \\ BCC502 - Metodologia Científica para Ciência da Computação \\\textbf{ Ciência como conhecimento derivado dos fatos da experiência}}
\author{Prof. Rodrigo Silva}
\date{}


\begin{document}

\maketitle

\section{What is science?}

\begin{enumerate}
    \item What is science?
    \begin{enumerate}
        \item Seems simple.
        \item Physics, Chemestry, Biology are.
        \item Music, art, theology are not.
        \item That is not it?
        \begin{enumerate}
            \item What is the common feature all the things on this list share?
            \item What it is that makes something a science? 
        \end{enumerate}
        \item One could say that science is just an attempt to understand, explain and predict the world.
        \item Isn't that what religion tries to do?
    \end{enumerate} 
    \item Is science a particular set methods?
    \begin{enumerate}
         \item Use of experiment
         \begin{enumerate}
            \item Astronomers cannot do experiments with planets, stars, black holes. 
         \end{enumerate}
    \end{enumerate}
    \item Philosophy of science
    \begin{enumerate}
        \item What is science?
        \item Why experimentation, observation , and thory construction have enabled scientists to unravel so many of the natures secrets?
        \item Should we trust science? Why?
    \end{enumerate}
    \item Origens da Ciência moderna
    \item 


    
\end{enumerate}

%\bibliographystyle{plain}
%\bibliography{references}
\end{document}

