\documentclass{article}
\usepackage[utf8]{inputenc}
\usepackage[margin=1.2in]{geometry}
\usepackage{hyperref}
\usepackage{listings}
\usepackage{xcolor}
\usepackage{natbib}
\usepackage{graphicx}
\usepackage{amsmath}

\title{\vspace{-2 cm}Universidade Federal de Ouro Preto \\ Inteligência Artificial \\ Estudo Dirigido 1}
\author{Prof. Rodrigo Silva}
\date{}


\begin{document}

\maketitle

\section{Introdução}

\begin{enumerate}
    \item A ciência é altalmente considerada
    \begin{enumerate}
        \item Existe uma crença amplamente aceita de que há algo especial sobre especial
        \begin{enumerate}
            \item Dizer que algo é científico parece dar credibilidade
            \item ``Os cientistas dizem ...''
            \item Científicamente comprovado
        \end{enumerate}
        \item Insinuação que a afirmação é particularmente bem fundamentada e que talvez esteja além da contestação.
        \item Qual a base para tal autoridade? 
    \end{enumerate}
    \item O que é especial em relação à ciência? O que vem a ser este ``método científico'' que leva a resultados especialmente meritórios e confiáveis?
    \begin{enumerate}
        \item A filosofia da ciência tenta responder estas perguntas
        \item A história da ciência coloca muitos problemas para os filósofos. 
        \item Galileu, Newton, Darwin, Eisntein chegaram às suas contribuições científicas por caminhos e métodos muito diferentes.
    \end{enumerate}
    \item Existem críticos, e.g., Paul Feyerbend - Contra o método
    \begin{enumerate}
        \item A ciência é uma religião moderma. 
        \item Não possui características especiais que a façam superior a nada.
        \item Sugere que a escolha entre teorias se reduz a opções determinadas por valores subjetivos e e desejos pessoais 
    \end{enumerate}
    \item Positivismo lógigo
    \begin{enumerate}
        \item Afirma que apenas as declarações verificáveis através da observação direta ou da prova lógica são significativas.
        \item Conflito com a física quantíca e o com o relativismo de Eisntein.
    \end{enumerate}
    \item \textit{Começaremos confusos e terminaremos confusos num nível mais alto.}
\end{enumerate}

\section{Indutivismo}

\begin{enumerate}
    \item 
\end{enumerate}

\section*{Fonte}

\begin{itemize}
    \item Chalmers, A. F., and Fiker, R. (1993). O que é ciência afinal? (pp. 23-63). São Paulo: Brasiliense.
\end{itemize}

%\bibliographystyle{plain}
%\bibliography{references}
\end{document}

