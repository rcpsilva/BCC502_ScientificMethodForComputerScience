%%% Template para anotações de aula
%%% Feito por Daniel Campos com base no template de Willian Chamma que fez com base no template de  Mikhail Klassen

\documentclass[12pt,a4paper, brazil]{article}

%%%%%%% INFORMAÇÕES DO CABEÇALHO
\newcommand{\workingDate}{\textsc{\selectlanguage{portuguese}\today}}
\newcommand{\userName}{BCC502}
\newcommand{\institution}{UFOP}
\usepackage{researchdiary_png}


\begin{document}

\begin{center}
{\textbf {\huge Título}}\\[5mm]
{\large Autor: } \\[2mm]
{\large Orientador: } \\[5mm]
\today\\[5mm] %% se quiser colocar data
\end{center}


%\section*{Resumo}

\section{Introdução}

A Figura \ref{fig:logo_ufop} apresenta a logo da UFOP.

\begin{figure}[!ht]
    \centering
    \includegraphics[width=0.2\textwidth]{ufop-logo.jpg}
    \caption{Logo da ufop}
    \label{fig:logo_ufop}
\end{figure}

A Tabela \ref{tab:tabx} apresenta .....

\begin{table}[!ht]
    \centering
    \begin{tabular}{ |r|c|l| }
        \hline
         cell1 & cell2 & cell3mnmn \\
         \hline
         \hline 
         cell4 & cell5mnmnm & cell6 \\
         \hline
         cell7mnmn & cell8 & cell9 \\
         \hline   
        \end{tabular}
        \caption{Tabela X}
        \label{tab:tabx}
\end{table}


\cite{bonawitz2019towards}

\subsection{Objetivos}

    % Objetivo geral e objetivos especificos

    A Tabela \ref{tab:taby} apresenta ...... e a Tabela \ref{tab:tabx} apresenta ....

    \begin{table}[!ht]
        \centering
        \begin{tabular}{ |r|c|l| }
            \hline
             cell1 & cell2 & cell3mnmn \\
             \hline
             \hline 
             cell4 & cell5mnmnm & cell6 \\
             \hline
             cell7mnmn & cell8 & cell9 \\
             \hline   
            \end{tabular}
            \caption{Tabela Y}
            \label{tab:taby}
    \end{table}
    
\cite{WinNT}

\cite{salahdine2019social}

\section{Revisão Bibliográfica} \label{sec:revisao}


%%% as referências devem estar em formato bibTeX no arquivo referencias.bib
\printbibliography

\end{document}