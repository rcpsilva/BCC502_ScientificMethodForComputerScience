\documentclass{article}
\usepackage[utf8]{inputenc}
\usepackage[margin=1.2in]{geometry}
\usepackage[T1]{fontenc}                
\usepackage[utf8]{inputenc}             
\usepackage[english,portuguese]{babel}
\usepackage{tikz}
  \usetikzlibrary{shapes,arrows,fit,calc,positioning}
  \tikzset{box/.style={draw, diamond, thick, text centered, minimum height=0.5cm, minimum width=1cm}}
  \tikzset{line/.style={draw, thick, -latex'}}


\title{Princípios de aprendizado de máquina \\ Inteligência Artificial}
\author{Prof. Rodrigo Pedrosa}

\usepackage{natbib}
\usepackage{graphicx}
\usepackage{amsmath}
\usepackage{hyperref}

\begin{document}

\maketitle

\

\section{Questões}

\begin{enumerate}

    \item Quais foram os erros cometidos pelo ``aluno de mestrado'' citado no capítulo 1?
    
    \item Quais são os estilos de pesquisa em computação?
    
    \item Desenvolvimento de software pode ser considerado pesquisa científica? Se sim, como este tipo de pesquisa deve ser conduzido?
    
    \item O que é uma hipótese? Por quê ela é importante?
    
    \item O que é o tema de pesquisa?
    
\end{enumerate}
    
%\bibliographystyle{plain}
%\bibliography{references}

\end{document}
