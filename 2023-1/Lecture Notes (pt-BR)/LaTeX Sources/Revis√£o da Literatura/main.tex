%%% Template para anotações de aula
%%% Feito por Daniel Campos com base no template de Willian Chamma que fez com base no template de  Mikhail Klassen



\documentclass[12pt,a4paper, brazil]{article}

%%%%%%% INFORMAÇÕES DO CABEÇALHO
\newcommand{\workingDate}{\textsc{\selectlanguage{portuguese}\today}}
\newcommand{\userName}{BCC502}
\newcommand{\institution}{UFOP}
\usepackage{researchdiary_png}




\begin{document}
\begin{center}
{\textbf {\huge Revisão da Literatura}}\\[5mm]
%{\large Autor: } \\[2mm]
%{\large Orientador: } \\[5mm]
\today\\[5mm] %% se quiser colocar data
\end{center}


%\section*{Resumo}

\section{Como fazer uma revisão de literatura?}

Fazer uma revisão da literatura é um processo importante para obter uma visão abrangente e atualizada sobre um determinado tema. Essa revisão envolve a coleta, análise e síntese de informações relevantes disponíveis na literatura acadêmica e científica. Aqui estão algumas etapas que você pode seguir para realizar uma revisão da literatura:

\begin{enumerate}

\item Defina seu objetivo: Determine claramente o objetivo da sua revisão da literatura. Isso ajudará a direcionar sua pesquisa e definir os critérios de inclusão para os estudos que você irá analisar.

\item Identifique as fontes de informação: Pesquise em bancos de dados acadêmicos, bibliotecas, revistas científicas, conferências e outras fontes confiáveis para encontrar estudos relevantes relacionados ao seu tema. Use palavras-chave e combinações de termos para melhorar sua pesquisa.

\item Selecione os estudos: Leia os títulos e resumos dos artigos encontrados e avalie sua relevância para o seu objetivo. Exclua os estudos que não atendem aos seus critérios de inclusão e selecione aqueles que são mais relevantes para sua revisão.

\item Leia e analise os estudos selecionados: Leia cuidadosamente os estudos selecionados e faça anotações sobre os principais pontos, métodos utilizados, resultados e conclusões. Identifique as semelhanças, diferenças e lacunas na literatura existente.

\item Organize a informação: Identifique os principais temas ou tópicos que emergem dos estudos analisados e organize suas descobertas de acordo com esses temas. Você pode usar tabelas, gráficos ou outras ferramentas visuais para ajudar a sintetizar e apresentar a informação de forma clara.

\item Analise e interprete os resultados: Faça uma análise crítica dos estudos revisados e interprete os resultados à luz do seu objetivo. Identifique padrões, contradições ou lacunas na literatura e discuta suas descobertas de forma clara e objetiva.

\item Escreva a revisão: Organize suas descobertas em uma estrutura lógica, que pode incluir introdução, metodologia, resultados e discussão. Certifique-se de fornecer referências bibliográficas adequadas para todos os estudos citados.

\item Revisão e edição: Revise e edite sua revisão da literatura, verificando a precisão, coesão e clareza do texto. Certifique-se de que suas citações e referências estejam corretas de acordo com o estilo de citação exigido.

\end{enumerate}

Lembre-se de que fazer uma revisão da literatura é um processo iterativo, e você pode precisar realizar várias rodadas de pesquisa e análise para obter uma revisão completa e atualizada. Também é importante manter-se atualizado sobre as últimas publicações na área de estudo, pois novos estudos podem surgir e ter impacto na sua revisão da literatura.

\section{Como definir o objetivo de uma revisão de literatura?}

Definir o objetivo de uma revisão da literatura é fundamental, pois ajuda a direcionar sua pesquisa e estabelecer os critérios de inclusão dos estudos que serão analisados. Aqui estão algumas etapas para ajudar a definir o objetivo da sua revisão de literatura:


\begin{enumerate}
    \item Escolha um tema: Comece selecionando um tema amplo ou específico de interesse para a sua revisão. Pode ser uma área de estudo específica, um problema de pesquisa, uma teoria, uma abordagem metodológica, uma intervenção ou qualquer outro tópico relevante.
    
    \item Refine o tema: Após escolher o tema, refine-o para torná-lo mais específico e delimitado. Isso ajudará a evitar que a revisão se torne muito abrangente e difícil de ser concluída. Considere limitar o período de tempo, o contexto geográfico, o público-alvo, as variáveis específicas ou outros aspectos relevantes para o seu tema.
    
    \item Faça perguntas de pesquisa: Formule perguntas de pesquisa que orientem sua revisão da literatura. Essas perguntas podem ser amplas ou específicas, dependendo do objetivo da sua revisão. Por exemplo, se você está revisando a eficácia de uma intervenção, uma pergunta poderia ser: "Quais são os estudos mais recentes que avaliam a eficácia da intervenção X em relação ao resultado Y?"
    
    \item Determine o tipo de revisão: Decida o tipo de revisão que você pretende realizar. Existem diferentes abordagens, como revisões sistemáticas, revisões narrativas, revisões scoping, revisões integrativas, entre outras. Cada tipo de revisão tem seus próprios objetivos e métodos específicos.
    
    \item Defina os critérios de inclusão e exclusão: Estabeleça critérios claros para selecionar os estudos que serão incluídos na sua revisão. Esses critérios podem incluir o tipo de estudo (por exemplo, ensaios clínicos randomizados, estudos observacionais), o idioma, o ano de publicação, a qualidade metodológica, a relevância para o seu objetivo, entre outros.
    
    \item Considere o público-alvo: Pense em quem será o público-alvo da sua revisão. Isso ajudará a definir o nível de detalhes técnicos e a linguagem apropriada para a sua revisão. Considere se a revisão é destinada a pesquisadores, profissionais da área, estudantes ou um público mais amplo.
    
    \item Ao definir o objetivo da sua revisão de literatura, lembre-se de que ele deve ser claro, específico e relevante para a área de estudo. Isso ajudará a guiar sua pesquisa, estruturação e análise dos estudos selecionados.

\end{enumerate}

Ao definir o objetivo da sua revisão de literatura, lembre-se de que ele deve ser claro, específico e relevante para a área de estudo. Isso ajudará a guiar sua pesquisa, estruturação e análise dos estudos selecionados.

\section{Por qual tipo de texto devo começar a minha revisão da literatura?}

Ao iniciar sua revisão da literatura, é útil começar com textos introdutórios e de referência que fornecem uma visão geral do campo e estabelecem um contexto para sua revisão. Aqui estão alguns tipos de textos que você pode considerar começar:

\begin{enumerate}
    \item Livros-texto: Consultar livros-texto sobre o tema ou área de estudo é uma maneira eficaz de obter uma introdução abrangente e sistemática. Esses livros geralmente fornecem uma visão geral do campo, explicam conceitos-chave, teorias, metodologias e podem até fornecer uma revisão resumida da literatura existente.
    \item Revisões sistemáticas anteriores (Systematic Literature Review, Surveys, Reviews): Pesquise por revisões sistemáticas publicadas anteriormente sobre o tema de interesse. As revisões sistemáticas são estudos que sintetizam e resumem a literatura existente de forma rigorosa e abrangente. Elas podem fornecer um ponto de partida valioso para identificar estudos-chave e lacunas na literatura.
    \item Capítulos de livro: Muitas vezes, os capítulos de livro abordam tópicos específicos dentro de uma área de estudo e fornecem informações detalhadas e atualizadas sobre esses tópicos. Procure por capítulos de livro que sejam relevantes para o seu tema e que tenham sido escritos por especialistas no campo.
    \item Artigos de revisão narrativa: Os artigos de revisão narrativa são uma forma menos rigorosa de revisão da literatura, mas podem fornecer uma visão geral e uma perspectiva histórica sobre o tema. Eles geralmente são escritos por especialistas reconhecidos no campo e podem fornecer informações úteis sobre os principais conceitos e desenvolvimentos em uma área específica.
    \item Relatórios técnicos e documentos governamentais: Em certas áreas de estudo, relatórios técnicos e documentos governamentais podem conter informações valiosas e atualizadas sobre pesquisas e políticas relacionadas ao tema. Verifique se há relatórios técnicos ou documentos governamentais relevantes disponíveis, especialmente se o seu tema estiver relacionado a políticas públicas, saúde, meio ambiente ou outras áreas similares.
\end{enumerate}

Esses textos introdutórios e de referência podem ajudar a fornecer uma base sólida para a sua revisão, ajudando você a entender o contexto do campo, identificar conceitos-chave, teorias e pesquisas fundamentais. No entanto, lembre-se de que esses textos devem ser complementados por estudos primários e mais recentes encontrados em revistas científicas, conferências e outras fontes acadêmicas relevantes.

\section{Como fazer a leitura dos textos selecionados? Como organizar a informação?}


Ao ler os textos selecionados para sua revisão da literatura, é importante adotar uma abordagem estratégica e organizar as informações de maneira eficiente. Aqui estão algumas dicas para a leitura e organização dos textos:


\begin{enumerate}
    
\item Antes de mergulhar na leitura completa, faça uma leitura rápida do título, resumo, introdução e conclusão do artigo. Isso lhe dará uma ideia geral do conteúdo e ajudará a decidir se o texto é relevante para sua revisão.

\item Anote informações importantes: Enquanto lê, faça anotações sobre os principais pontos, conceitos, teorias, metodologias e resultados mencionados no texto. Destaque trechos relevantes, citações importantes e estatísticas relevantes. Mantenha um registro organizado dessas informações para referência futura.

\item Utilize técnicas de leitura ativa: Adote uma abordagem de leitura ativa, fazendo perguntas enquanto lê, relacionando o texto a outros estudos e refletindo sobre suas implicações. Isso ajudará você a compreender melhor o conteúdo e a desenvolver uma visão crítica sobre os argumentos apresentados.

\item Identifique padrões e tendências: Ao ler vários textos, procure por padrões, convergências ou divergências nos resultados, opiniões ou perspectivas dos autores. Identifique tendências emergentes ou lacunas na literatura. Isso ajudará a criar uma estrutura coerente para a sua revisão.

\item Utilize ferramentas de organização: Existem várias ferramentas que podem auxiliar na organização das informações, como softwares de gestão de referências bibliográficas (por exemplo, Zotero, Mendeley, EndNote), aplicativos de anotações ou simplesmente um sistema de arquivamento físico ou digital. Essas ferramentas permitem armazenar e categorizar os artigos, adicionar notas e tags relevantes, facilitando a recuperação das informações durante a redação da revisão.

\item Crie um esquema ou mapa conceitual: À medida que você lê e analisa os textos selecionados, pode ser útil criar um esquema ou mapa conceitual para visualizar as relações entre os diferentes conceitos, teorias e resultados discutidos nos artigos. Isso ajudará a organizar as informações de forma clara e a identificar lacunas ou áreas que precisam de mais pesquisa.

\item Mantenha um registro atualizado das fontes: Certifique-se de manter um registro atualizado de todas as fontes que você leu, incluindo os detalhes bibliográficos completos. Isso facilitará a citação correta das referências durante a redação da sua revisão.

Ao ler e organizar os textos selecionados, é importante manter um equilíbrio entre uma compreensão aprofundada do conteúdo e a eficiência na extração das informações relevantes. Planeje seu tempo de leitura de forma adequada e estabeleça metas realistas para cada sessão de leitura.

\end{enumerate}


\section{Modelo para organizar as informações coletadas na revisão}

Aqui está um modelo simples de tabela que você pode usar para organizar as informações coletadas durante a revisão da literatura:


\begin{table}[ht]
    \centering
    \begin{tabular}{|p{2.8cm}|p{2.5cm}|p{2.5cm}|p{2.5cm}|p{3cm}|}
    \hline
    \textbf{Referência Bibliográfica} & \textbf{Objetivo do Estudo} & \textbf{Metodologia} & \textbf{Principais Resultados} & \textbf{Conclusões (Implicações)} \\
    \hline
    [Referência 1] & & & & \\
    \hline
    [Referência 2] & & & & \\
    \hline
    [Referência 3] & & & & \\
    \hline
    ... & & & & \\
    \hline
    \end{tabular}
\end{table}

Aqui está uma breve descrição de cada coluna da tabela:

\begin{enumerate}
    \item Referência Bibliográfica: Inclua aqui as informações completas da referência bibliográfica do artigo ou texto que você está revisando, incluindo o autor(es), título, nome da revista (ou nome do livro), ano de publicação, volume, número e páginas.

    \item Objetivo do Estudo: Resuma de forma concisa o objetivo do estudo ou a pergunta de pesquisa que o artigo aborda. Isso ajudará a identificar rapidamente o foco de cada estudo revisado.

    \item Metodologia: Descreva brevemente a metodologia utilizada no estudo, destacando aspectos relevantes, como o tipo de estudo (experimental, observacional, revisão sistemática, etc.), a amostra, as técnicas de coleta de dados e análise.

    \item Principais Resultados: Liste de forma resumida os principais resultados ou descobertas do estudo. Inclua estatísticas relevantes, efeitos observados, descobertas-chave ou conclusões principais.

    \item Conclusões/Implicações: Indique as conclusões ou implicações do estudo em relação à sua área de pesquisa. Considere como os resultados se relacionam com outros estudos revisados e se eles apoiam ou contradizem as teorias existentes.
\end{enumerate}

Essa tabela é apenas um modelo básico e pode ser personalizada de acordo com as necessidades específicas da sua revisão. Você pode adicionar mais colunas, como "Análise de Lacunas" para identificar áreas que precisam de mais pesquisa, ou "Pontos Fortes e Limitações" para avaliar criticamente cada estudo revisado.

Lembre-se de preencher a tabela à medida que avança na leitura dos artigos selecionados, mantendo-a atualizada e organizada. Isso facilitará a revisão das informações durante o processo de escrita e ajudará você a identificar padrões, tendências e lacunas na literatura revisada.

%%% as referências devem estar em formato bibTeX no arquivo referencias.bib
\printbibliography

\end{document}