%%% Template para anotações de aula
%%% Feito por Daniel Campos com base no template de Willian Chamma que fez com base no template de  Mikhail Klassen



\documentclass[12pt,a4paper, brazil]{article}

%%%%%%% INFORMAÇÕES DO CABEÇALHO
\newcommand{\workingDate}{\textsc{\selectlanguage{portuguese}\today}}
\newcommand{\userName}{BCC502}
\newcommand{\institution}{UFOP}
\usepackage{researchdiary_png}




\begin{document}
\begin{center}
{\textbf {\huge Fundamentos}}\\[5mm]
%{\large Autor: } \\[2mm]
%{\large Orientador: } \\[5mm]
\today\\[5mm] %% se quiser colocar data
\end{center}


%\section*{Resumo}

\section{O que é a seção de fundamentos?}

Em uma tese de graduação em ciência da computação, a seção de fundamentos fornece uma base de conceitos fundamentais, teorias e metodologias que são essenciais para compreender os aspectos técnicos de sua pesquisa. 

\section{Como escrever uma boa seção de fundamentos?}

Escrever uma seção de fundamentos agradável requer clareza, organização e comunicação efetiva dos conceitos essenciais e informações de base. Aqui estão algumas dicas para ajudá-lo a escrever uma seção de fundamentos forte:

\begin{enumerate}
    \item Entenda seu público: Considere o conhecimento e a experiência prévia de seus leitores. Adapte suas explicações de acordo, evitando jargões desnecessários e fornecendo informações suficientes para uma compreensão abrangente.

    \item Esboce a seção: Crie uma estrutura clara e lógica para a seção de fundamentos. Esboce os principais tópicos que você precisa abordar e organize-os de maneira que fluam suavemente de um conceito para o próximo.

    \item Comece com o básico: Comece introduzindo os conceitos e teorias fundamentais relacionados à sua pesquisa. Comece com os conceitos mais básicos e construa gradualmente sobre eles, garantindo que cada conceito seja explicado antes de passar para o próximo.

    \item Forneça definições claras: Defina quaisquer termos técnicos ou conceitos que possam ser desconhecidos para seus leitores. Evite presumir conhecimento prévio e forneça definições concisas, porém abrangentes, para estabelecer uma compreensão comum.

    \item Use exemplos e ilustrações: Aumente a clareza de suas explicações fornecendo exemplos e ilustrações concretas. Isso ajuda os leitores a visualizar e compreender ideias complexas. Recursos visuais, como diagramas ou figuras, podem ser especialmente úteis para explicar conceitos técnicos.

    \item Relacione conceitos à sua pesquisa: Conecte continuamente os conceitos fundamentais ao contexto específico de sua pesquisa. Explique por que cada conceito é relevante e como ele serve de base para as seções subsequentes de sua tese.

    \item Use linguagem concisa e precisa: Seja claro e conciso em sua escrita. Evite prolixidade desnecessária e use uma linguagem precisa para transmitir suas ideias. Divida conceitos complexos em partes menores e digeríveis para torná-los mais acessíveis.

    \item Forneça referências: Sempre que introduzir um conceito ou método, forneça referências apropriadas para respaldar suas afirmações e reconhecer os autores originais. Isso adiciona credibilidade ao seu trabalho e permite que os leitores explorem as fontes para obter uma compreensão mais aprofundada.
\end{enumerate}

\section{Dica}

Neste estágio, no qual vocês ainda têm pouca experiência com o tema de pesquisa escolhido, pode ser útil perguntar ao ChatGPT: \texttt{``What should go in a fundamentals section of an undergrad thesis regarding [seu problema de pesquisa]?''} Vocês podem tentar o \textit{prompt} em português mas, me parece que em inglês, resultados melhores são obtidos.

%%% as referências devem estar em formato bibTeX no arquivo referencias.bib
\printbibliography

\end{document}