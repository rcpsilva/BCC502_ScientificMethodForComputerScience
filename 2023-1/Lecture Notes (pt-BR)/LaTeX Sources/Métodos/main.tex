%%% Template para anotações de aula
%%% Feito por Daniel Campos com base no template de Willian Chamma que fez com base no template de  Mikhail Klassen



\documentclass[12pt,a4paper, brazil]{article}

%%%%%%% INFORMAÇÕES DO CABEÇALHO
\newcommand{\workingDate}{\textsc{\selectlanguage{portuguese}\today}}
\newcommand{\userName}{BCC502}
\newcommand{\institution}{UFOP}
\usepackage{researchdiary_png}

\begin{document}
\begin{center}
{\textbf {\huge Métodos}}\\[5mm]
%{\large Autor: } \\[2mm]
%{\large Orientador: } \\[5mm]
\today\\[5mm] %% se quiser colocar data
\end{center}


%\section*{Resumo}

\section{Como escrever o Métodos?}

Um capítulo de ``Métodos'' em uma monografia de Ciência da Computação descreve os recursos, ferramentas e procedimentos utilizados para realizar a pesquisa ou implementar o projeto. Essa seção fornece uma descrição detalhada dos materiais utilizados e dos métodos adotados, permitindo que outros pesquisadores possam reproduzir o estudo ou projeto com base nessas informações. 


Para enfatizar a contribuição do autor, é comum que o método de implementação ou abordagem de pesquisa seja escrito em um capítulo separado, com o nome da abordagem proposta, enquanto outro capítulo é dedicado ao método experimental, que descreve como a abordagem proposta foi validada. Essa separação permite uma melhor organização do conteúdo e facilita a compreensão dos leitores sobre a implementação e validação da abordagem. Então se o seu trabalho propõe uma vova abordagem ou algoritmo, faça a separação. 

\section{O que deve conter o capítulo destinado à abordagem proposta?}

No capítulo da abordagem proposta, é importante descrever em detalhes a metodologia adotada para implementar o sistema, desenvolver o software ou realizar a pesquisa. Explique o modelo teórico, algoritmos ou técnicas específicas utilizadas e justifique suas escolhas. Se necessário, inclua diagramas, pseudocódigos ou fluxogramas para auxiliar na compreensão da abordagem proposta.

\section{O que deve conter o capítulo destinado ao método experimental?}

\begin{enumerate}
    \item Descrição dos materiais: Liste todos os recursos, ferramentas, software, hardware e/ou datasets específicos utilizados em sua pesquisa ou implementação. Inclua detalhes como versões do software, especificações técnicas do hardware, fontes dos datasets, entre outros. 

    \item Método de coleta de dados: Explique como os dados foram obtidos. Descreva os procedimentos utilizados para coletar informações relevantes para a pesquisa, como entrevistas, questionários, observações, experimentos, coleta de dados em campo ou extração de dados de fontes secundárias.
    
    \item Procedimentos experimentais: Se a sua pesquisa envolve experimentação, explique como os experimentos foram conduzidos. Descreva o design experimental, as variáveis controladas, as configurações e condições utilizadas, os critérios de amostragem e os métodos de análise de dados.

    \item Avaliação e métricas: Se o seu trabalho envolve avaliação de desempenho ou comparação com outros métodos, explique as métricas utilizadas e como elas foram calculadas. Descreva também os procedimentos de validação adotados para garantir a precisão e confiabilidade dos resultados.
    
    \item Considerações éticas: Se a sua pesquisa envolve seres humanos, animais ou dados sensíveis, mencione as considerações éticas e as aprovações éticas obtidas, conforme necessário. Certifique-se de seguir os padrões éticos estabelecidos pela sua instituição ou pela comunidade científica.
    
    \item Limitações: Discuta as limitações do seu estudo ou implementação. Reconheça e descreva quaisquer restrições, restrições de tempo, viéses potenciais ou outras limitações que possam afetar os resultados ou interpretações.

\end{enumerate}

Lembre-se de que a organização e a estrutura da seção de "Materiais e Métodos" podem variar de acordo com as necessidades do trabalho ou diretrizes específicas fornnecidas pelo orientador.




%%% as referências devem estar em formato bibTeX no arquivo referencias.bib
\printbibliography

\end{document}