%%% Template para anotações de aula
%%% Feito por Daniel Campos com base no template de Willian Chamma que fez com base no template de  Mikhail Klassen



\documentclass[12pt,a4paper, brazil]{article}

%%%%%%% INFORMAÇÕES DO CABEÇALHO
\newcommand{\workingDate}{\textsc{\selectlanguage{portuguese}\today}}
\newcommand{\userName}{BCC502}
\newcommand{\institution}{UFOP}
\usepackage{researchdiary_png}


\begin{document}

\begin{center}
{\textbf {\huge Método}}\\[5mm]
{\large Critérios de Correção } \\[2mm]
{\large Prof. Rodrigo Silva } \\[5mm]
\today\\[5mm] %% se quiser colocar data
\end{center}


%\section*{Resumo}

\section*{Critérios}

\begin{enumerate}
    \item \textbf{Organização geral do texto:}
        \begin{enumerate}
            \item (0.5 pt) Apresenta um texto claro
            \item (0.5 pt) Faz conexão entre as seções
        \end{enumerate}
    \item \textbf{Título:} 
    \begin{enumerate}
        \item (0.5 pt) Apresenta título condizente com o problema de pesquisa e com a revisão da literatura.
    \end{enumerate}
    \item \textbf{Introdução:} 
        \begin{enumerate}
            \item (0.5 pt) Definição precisa do problema com base na literatura.
            \item (0.5 pt) Motivação (Por quê este problema é relevante?)
            \item (0.5 pt) Definição das perguntas de pesquisa. O que você está tentando descobrir/entender? 
        \end{enumerate}
    \item \textbf{Revisão da literatura:} 
        \begin{enumerate}
            \item (0.5 pt) Indentifica os principais temas ou tópicos que emergem dos estudos revisados.
            \item (0.5 pt) Apresenta análise crítica dos estudos revisados à luz do problema de pesquisa escolhido.
        \end{enumerate}
    \item \textbf{Fundamentos}:
        \begin{enumerate}
            \item (0.5 pt) Fornece definições claras de termos e métodos importantes para a pesquisa.
            \item (0.5 pt) Relaciona os conceitos/métodos apresentados com o problema de pesquisa.  
        \end{enumerate}
    \item \textbf{Método}
        \begin{enumerate}
            \item (1.5 pt) Descrição dos materiais e método proposto.
            \item (1.5 pt) Descrição da coleta de dados e procedimentos experimentais.
            \item (1 pt) Definição das métricas de avaliação. 
            \item (1 pt) Discução das limitações.
        \end{enumerate}
    \item \textbf{Referências}
        \begin{enumerate}
            \item (0.5 pt) Apresenta as referência no padrão adequado. 
        \end{enumerate}
\end{enumerate}


% A good survey must include
    % 1 - Taxonomy
    % 2 - Critical view of the literature
    % 3 -  Identification of the key objectives in the area and the main ways in which they have been addressed
    % 4 - Discussion on related application areas 
    % 5 - Comparative analysis, limitations, drawbacks, misleading opinions expressed in the existing literature.
    % 6 - well-grounded conclusions as to the current progress and possible future prospects and promising directions
    
% A good survey should NOT be
    % 1- A collection of abstracts of the papers without any critical assessment.
    % 2- A collection of graphs, pie plots, and others displaying general statistics and tendencies in the area but not being supported by any in-depth analysis.
    % 3 - A very verbally biased presentation of the material lacking any depth of exposure.

% Hints for preparing a good survey
    % 1 - A survey can also be good with a lower number of discussed papers, for example with a focus on key papers or most cited papers.
    % 2 - Clarify the used bibliometric and retrieval method
    % 3 - Decide for one way among the possible ways of presenting results (like chronological, in historical context, by method grouping, by problem classes, by used paradigms) and follow this line.

\printbibliography

\end{document}