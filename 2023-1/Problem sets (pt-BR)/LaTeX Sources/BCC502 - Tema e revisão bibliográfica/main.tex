\documentclass{article}
\usepackage[utf8]{inputenc}
\usepackage[margin=1.2in]{geometry}
\usepackage[T1]{fontenc}                
\usepackage[utf8]{inputenc}             
\usepackage[english,portuguese]{babel}
\usepackage{tikz}
  \usetikzlibrary{shapes,arrows,fit,calc,positioning}
  \tikzset{box/.style={draw, diamond, thick, text centered, minimum height=0.5cm, minimum width=1cm}}
  \tikzset{line/.style={draw, thick, -latex'}}


\title{BCC502 - Metodologia Científica\\
Tema e Problema de Pesquisa}
\author{Prof. Rodrigo Pedrosa}

\usepackage{natbib}
\usepackage{graphicx}
\usepackage{amsmath}
\usepackage{hyperref}

\begin{document}

\maketitle

\section{Leitura recomendada}

Capítulos 3 - Wazlawick, Raul Sidnei. Metodologia de pesquisa para ciência da computação. Vol. 1. Elsevier, 2009.


\section{Questões}

\begin{enumerate}

    \item O que é o tema de pesquisa?
    \item Qual a diferença entre o tema e o objetivo de pesquisa?
    \item Quais objetivos \textbf{não} são considerados bons objetivos? Explique.
    \item Quais são os componentes da descrição do problema de pesquisa?
    \item Quais são os passos para a escolha de um objetivo de pesquisa?
    \item Como deve ser feita a revisão de literatura?
     
\end{enumerate}
    
%\bibliographystyle{plain}
%\bibliography{references}

\end{document}
