\documentclass{article}
\usepackage[utf8]{inputenc}
\usepackage[margin=1.2in]{geometry}
\usepackage{hyperref}
\usepackage{listings}
\usepackage{xcolor}
\usepackage{natbib}
\usepackage{graphicx}
\usepackage{amsmath}

\title{\vspace{-2cm}Universidade Federal de Ouro Preto \\ Objetivos e Precedimento Metodológico}
\author{Prof. Rodrigo Silva}
\date{}

\begin{document}

\maketitle

\section*{Instruções}

Neste ponto espera-se que você já tenha um problema uma hipótese de pesquisa bem definidos.

Nesta estapa você deve apresentar os objetivos e o procedimento metodológico que você pretende seguir para testar a hipótese de pesquisa. Leiam o capítulo 6 do livro texto para auxiliá-los. Ler o capítulo 9 também pode ser útil.

Nesta entrega, cada aluno deverá apresentar:
\begin{enumerate}
    \item Problema e contextualização do problema (melhorar o que foi feito na primeira entrega ou refazê-lo);
    \item Hipótese de pesquisa e justificativa;
    \item Objetivos geral e específicos;
    \item Procedimento metodológico.
\end{enumerate}

Note que:
\begin{itemize}
    \item É obrigatória a utilização do template disponível em \url{https://github.com/rcpsilva/BCC502_ScientificMethodForComputerScience/blob/main/2024-2/Template%20Monografia%20-%20BCC502.zip}.
    \item É obrigatório o uso de BibTeX para as referências.
    \item O PDF, e somente o PDF, deve ser submetido pelo Moodle.
\end{itemize}

%\bibliographystyle{plain}
%\bibliography{references}
\end{document}