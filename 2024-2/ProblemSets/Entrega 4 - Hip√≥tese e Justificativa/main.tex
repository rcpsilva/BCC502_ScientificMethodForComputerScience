\documentclass{article}
\usepackage[utf8]{inputenc}
\usepackage[margin=1.2in]{geometry}
\usepackage{hyperref}
\usepackage{listings}
\usepackage{xcolor}
\usepackage{natbib}
\usepackage{graphicx}
\usepackage{amsmath}

\title{\vspace{-2 cm}Universidade Federal de Ouro Preto \\ Definição do Problema e Contextualização}
\author{Prof. Rodrigo Silva}
\date{}


\begin{document}

\maketitle

\section*{Instruções}

\begin{enumerate}
    \item Nesta entrega cada aluno deverá definir o problema que será tratado em sus respectiva pesquisa apresentando também a sua Contextualização.
    \item O problema escolhido deve pertencer ao tema do grupo.
    \item O problema deve ser definido com precisão. Se possível, apresentar uma definição formal/matemática para o problem. 
    \item Se tal definição já existir na literatura, você pode repetir a definição. Obviamente, deve-se fornecer a referência da fonte original. 
    \item A contextualização deve demonstrar a importância de de se resolver o problema escolhindo com referências diretas à literatura. 
    \item É obrigatória a utilização do template disponível em \url{https://github.com/rcpsilva/BCC502_ScientificMethodForComputerScience/blob/main/2024-2/Template%20Monografia%20-%20BCC502.zip}
    \item É obrigatório o uso de bibtex para as referências. 
    \item O PDF, e APENAS o PDF, devem ser submetidos pelo moodle.  
\end{enumerate}    




%\bibliographystyle{plain}
%\bibliography{references}
\end{document}

