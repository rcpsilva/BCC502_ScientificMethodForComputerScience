\documentclass{article}
\usepackage[utf8]{inputenc}
\usepackage[margin=1.2in]{geometry}
\usepackage{hyperref}
\usepackage{listings}
\usepackage{xcolor}
\usepackage{natbib}
\usepackage{graphicx}
\usepackage{amsmath}

\title{\vspace{-2cm}Universidade Federal de Ouro Preto \\ Definição do Problema e Contextualização}
\author{Prof. Rodrigo Silva}
\date{}

\begin{document}

\maketitle

\section*{Instruções}

Na primeira entrega, vocês definiram um problema e explicaram por que é importante resolvê-lo. Agora, vocês já devem ter lido alguns trabalhos que propõem soluções para esse problema ou para algum problema semelhante. Nesta etapa, vocês precisam definir uma hipótese de pesquisa. Em geral, essa hipótese costuma ser formulada de forma que, se demonstrada verdadeira, indique que a sua proposta para solucionar o problema é superior às já existentes. Também é possível definir uma hipótese que busque demonstrar que algum método já proposto é melhor do que outros para resolver o problema.

Além de definir a hipótese, vocês precisam justificá-la, ou seja, convencer o leitor—com argumentos baseados na teoria e na literatura relacionada—de que essa hipótese tem grande chance de ser verdadeira.

Assim, nesta entrega, cada aluno deverá apresentar:
\begin{enumerate}
    \item Problema e contextualização do problema (melhorar o que foi feito na primeira entrega ou refazê-lo);
    \item Hipótese de pesquisa e justificativa.
\end{enumerate}

Note que:
\begin{enumerate}
    \item O problema deve ser definido com precisão. Se possível, apresente uma definição formal/matemática para o problema. Se tal definição já existir na literatura, você pode repeti-la, desde que forneça a referência da fonte original.
    \item A contextualização deve demonstrar a importância de se resolver o problema escolhido, com referências diretas à literatura.
    \item A hipótese deve ser original, ou seja, não deve ser uma hipótese já testada na literatura.
    \item A justificativa da hipótese deve ser baseada na literatura relacionada e/ou na teoria do assunto.
    \item É obrigatória a utilização do template disponível em \url{https://github.com/rcpsilva/BCC502_ScientificMethodForComputerScience/blob/main/2024-2/Template%20Monografia%20-%20BCC502.zip}.
    \item É obrigatório o uso de BibTeX para as referências.
    \item O PDF, e somente o PDF, deve ser submetido pelo Moodle.
\end{enumerate}

%\bibliographystyle{plain}
%\bibliography{references}
\end{document}